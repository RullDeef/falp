\chapter{Практические задания}

\section{Задание 1}

Написать хвостовую рекурсивную функцию \texttt{my-reverse}, которая развернет верхний
уровень своего списка-аргумента \texttt{lst}.

\begin{lstlisting}[language=Lisp]
(defun my-reverse (lst)
	(funcall #'(lambda (fn) (funcall fn fn lst nil))
		(lambda (self lst res)
			(cond ((null lst) res)
				(t (funcall self self (cdr lst) (cons (car lst) res)))))))
\end{lstlisting}

\section{Задание 2}

Написать функцию, которая возвращает первый элемент списка-аргумента, который сам является непустым списком.

\begin{lstlisting}[language=Lisp]
(defun get-first-list (lst)
	(cond ((null lst) nil)
		((and (listp (car lst)) (not (null (car lst))))
			(car lst))
		(t (get-first-list (cdr lst)))))
\end{lstlisting}

\section{Задание 3}

Написать функцию, которая выбирает из заданного списка только те числа, которые больше 1 и меньше 10.
(Вариант: между двумя заданными границами).

\begin{lstlisting}[language=Lisp]
(defun num-between (num low high)
	(and (< low num) (< num high)))
	
(defun select-between (lst low high)
	(cond
		((null lst) nil)
		((and (< low (car lst)) (< (car lst) high))
			(cons (car lst) (select-between (cdr lst) low high)))
		(t (select-between (cdr lst) low high))))
\end{lstlisting}

\clearpage

\section{Задание 4}

Напишите рекурсивную функцию, которая умножает на заданное число-аргумент все
числа из заданного списка-аргумента, когда

\begin{enumerate}[a)]
	\item все элементы списка -- числа,
	\item элементы списка -- любые объекты.
\end{enumerate}

\begin{lstlisting}[language=Lisp]
(defun mult-a (num lst)
	(cond
		((null lst) nil)
		(t (cons ( * num (car lst)) (mult-a num (cdr lst))))))

(defun mult-b (num lst)
	(cond
		((null lst) nil)
		((numberp (car lst))
			(cons ( * num (car lst)) (mult-b num (cdr lst))))
		(t (cons (car lst) (mult-b num (cdr lst))))))
\end{lstlisting}

\section{Задание 5}

Напишите функцию, \texttt{select-between}, которая из списка-аргумента, содержащего только числа, выбирает только те, которые расположены между двумя указанными границами-аргументами и возвращает их в виде списка (упорядоченного по возрастанию списка чисел
(+2 балла)).

\begin{lstlisting}[language=Lisp]
(defun num-between (num low high)
	(and (< low num) (< num high)))
	
(defun select-between (lst low high)
	(cond
		((null lst) nil)
		((and (< low (car lst)) (< (car lst) high))
			(cons (car lst) (select-between (cdr lst) low high)))
		(t (select-between (cdr lst) low high))))
\end{lstlisting}

\clearpage

\section{Задание 6}

Написать рекурсивную версию (с именем \texttt{rec-add}) вычисления суммы чисел заданного списка:

\begin{enumerate}[a)]
	\item одноуровнего смешанного,
	\item структурированного.
\end{enumerate}

\begin{lstlisting}[language=Lisp]
(defun __rec-add-a (lst sum)
	(cond
		((null lst) sum)
		((numberp (car lst))
			(__rec-add-a (cdr lst) (+ sum (car lst))))
		(t (__rec-add-a (cdr lst) sum))))

(defun rec-add-a (lst)
	(__rec-add-a lst 0))

(defun __rec-add-b (lst sum)
	(cond
		((null lst) sum)
		((numberp (car lst))
			(__rec-add-b (cdr lst) (+ sum (car lst))))
		((listp (car lst))
			(__rec-add-b (cdr lst) (+ sum (rec-add-b (car lst)))))
		(t (__rec-add-b (cdr lst) sum))))

(defun rec-add-b (lst)
	(__rec-add-b lst 0))
\end{lstlisting}

\section{Задание 7}

Написать рекурсивную версию с именем \texttt{recnth} функции \texttt{nth}.

\begin{lstlisting}[language=Lisp]
(defun recnth (index lst)
	(cond
		((null lst) nil)
		((eql index 0) (car lst))
		(t (recnth (- index 1) (cdr lst)))))
\end{lstlisting}

\clearpage

\section{Задание 8}

Написать рекурсивную функцию \texttt{allodd}, которая возвращает \texttt{t} когда все элементы списка
нечетные.

\begin{lstlisting}[language=Lisp]
(defun allodd (lst)
	(cond
		((null lst) t)
		((evenp (car lst)) nil)
		(t (allodd (cdr lst)))))
\end{lstlisting}

\section{Задание 9}

Написать рекурсивную функцию, которая возвращает первое нечетное число из списка (структурированного), возможно создавая некоторые вспомогательные функции.

\begin{lstlisting}[language=Lisp]
(defun get-first-odd (lst)
	(cond
		((null lst) nil)
		((and (numberp (car lst)) (oddp (car lst)))
			(car lst))
		((listp (car lst))
			(or (get-first-odd (car lst))
				(get-first-odd (cdr lst))))
		(t (get-first-odd (cdr lst)))))
\end{lstlisting}

\section{Задание 10}

Используя cons-дополняемую рекурсию с одним тестом завершения,
написать функцию которая получает как аргумент список чисел, а возвращает список
квадратов этих чисел в том же порядке.

\begin{lstlisting}[language=Lisp]
(defun squares (lst)
	(if (null lst) nil
		(cons ( * (car lst) (car lst))
			(squares (cdr lst)))))
\end{lstlisting}
