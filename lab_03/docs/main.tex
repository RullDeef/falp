\chapter{Практические задания}

\section{Задание 1}

Написать функцию, которая принимает целое число и возвращает первое
четное число, не меньшее аргумента.

\begin{lstlisting}[language=Lisp]
; вариант 1
(defun first-even-ge (num)
	(if (= (mod num 2) 0)
		num
		(+ num 1)
	)
)

; вариант 2
(defun first-even-ge (num)
	(+ num (mod num 2))
)
\end{lstlisting}

\section{Задание 2}

Написать функцию, которая принимает число и возвращает число
того же знака, но с модулем на 1 больше модуля аргумента.

\begin{lstlisting}[language=Lisp]
; вариант 1
(defun plus-mod (num)
	(if (< num 0)
		(- num 1)
		(+ num 1)
	)
)

; вариант 2
(defun plus-mod (num)
	((if (< num 0) - +) num 1)
)
\end{lstlisting}

\clearpage

\section{Задание 3}

Написать функцию, которая принимает два числа и возвращает
список из этих чисел, расположенный по возрастанию.

\begin{lstlisting}[language=Lisp]
(defun sorted-2list (a b)
	(if (< a b)
		(list a b)
		(list b a)
	)
)
\end{lstlisting}

\section{Задание 4}

Написать функцию, которая принимает три числа и возвращает \texttt{T} только
тогда, когда первое число расположено между вторым и третьим.

\begin{lstlisting}[language=Lisp]
(defun betweenp (a b c)
	(or (and (< b a) (< a c))
		(and (< c a) (< a b))
	)
)
\end{lstlisting}

\section{Задание 5}

Каков результат вычисления следующих выражений?

\begin{table}
	\centering
	\begin{tabular}{|l|l|}
		\hline
		\textbf{Выражение} & \textbf{Результат} \\
		\hline
		\texttt{(and 'fee 'fie 'foe)} & \texttt{FOE} \\
		\hline
		\texttt{(or nil 'fie 'foe)} & \texttt{FIE} \\
		\hline
		\texttt{(and (equal 'abc 'abc) 'yes)} & \texttt{YES} \\
		\hline
		\texttt{(or 'fee 'fie 'foe)} & \texttt{FEE} \\
		\hline
		\texttt{(and nil 'fie 'foe)} & \texttt{NIL} \\
		\hline
		\texttt{(or (equal 'abc 'abc) 'yes)} & \texttt{T} \\
		\hline
	\end{tabular}
\end{table}

\section{Задание 6}

Написать предикат, который принимает два числа-аргумента и возвращает
\texttt{T}, если первое число не меньше второго.

\begin{lstlisting}[language=Lisp]
(defun gr-eq (a b)
	(>= a b)
)
\end{lstlisting}

\section{Задание 7}

Какой из следующих двух вариантов предиката ошибочен и почему?

\begin{enumerate}
	\item \texttt{(defun pred1 (x) (and (numberp x) (plusp x)))}
	\item \texttt{(defun pred2 (x) (and (plusp x) (numberp x)))}
\end{enumerate}

Второй вариант ошибочен, так как при вычислении формы \texttt{and} первым будет вычисляться предикат \texttt{plusp} и в случаях, когда аргумент \texttt{x} не будет являться числом, возникнет ошибка.

\section{Задание 8}

Решить задачу 4, используя для ее решения конструкции
\texttt{IF}, \texttt{COND}, \texttt{AND}/\texttt{OR}.

\begin{lstlisting}[language=Lisp]
; IF
(defun betweenp (a b c)
	(if (< b c)
		(if (< b a) (< a c))
		(if (< c a) (< a b))
	)
)

; COND
(defun betweenp (a b c)
	(cond
		((< b a) (< a c))
		((< c a) (< a b))
	)
)

; AND/OR
(defun betweenp (a b c)
	(or (and (< b a) (< a c))
		(and (< c a) (< a b))
	)
)
\end{lstlisting}

\clearpage

\section{Задание 9}

Переписать функцию \texttt{how-alike}, приведенную в лекции и использующую \texttt{COND}, используя
только конструкции \texttt{IF}, \texttt{AND}/\texttt{OR}.
