\Introduction

Программа на Lisp представляет собой вызов функции на верхнем уровне. Функции в
Lisp делятся на \textit{типичные} (математические) функции и \textit{формы} — функции, которые особым
образом обрабатывают свои аргументы, т. е. требуют специальной обработки. Кроме этого,
функции в Lisp носят \textit{частичный характер} т. е. по разному, иногда не корректно работают на
множестве S-выражений.

Синтаксически программа оформляется в виде \textit{S-выражения} (обычно -- списка). S-выражение, попавшее на вход системы анализирует функция \texttt{eval}. S-выражение очень часто
может быть \textit{структурированным}.

\textbf{Цель работы}: приобрести навыки работы в системе Common Lisp.

Для достижения поставленной цели необходимо выполнить следующие задачи:

\begin{itemize}[$\bullet$]
	\item изучить работу форм;
	\item изучить правила работы функций \texttt{cond}, \texttt{if}, \texttt{and}/\texttt{or} на различных списках-аргументах;
	\item изучить особенности работы форм в Lisp;
	\item проанализировать эффективность работы разных реализаций.
\end{itemize}
